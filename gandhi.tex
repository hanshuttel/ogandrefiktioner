I computerspillet \emph{Civilization}, hvor spilleren styrer en fiktiv civilisation fra en stenalderboplads til en rumfarende supermagt, figurerer der fiktive udgaver af virkelighedens historiske verdensledere, hvis adfærd bl.a. reguleres af talværdier - f.eks. er en verdensleders tilbøjelighed til at kaste sig ud i angrebskrige repræsenteret ved en \texttt{aggression}-værdi. Mange af spillets hændelser påvirker de forskellige lederes og kulturers attributter. F.eks. betød indførelsen af demokrati at lederens \texttt{aggression} reduceredes med $2$, for at simulere at demokratier er fredeligere end tyrannier.

En af spillets verdensledere er den indiske revolutionsleder og pacifist, Mohandas Gandhi. For at skildre Gandhi som en fredelig leder valgte spillets udviklere at give ham en \texttt{aggression} på $1$; den laveste hos nogen af spillets fiktionaliserede verdensledere. I hovedparten af spillet opførte Gandhi sig som man ville forvente: Han fokuserede på kulturel og videnskabelig udvikling, etablerede fredelige handelsforbindelser med nabolandene, og forholdt sig neutralt i andres krige. Men han havde også et temmelig uheldigt træk: I det øjeblik han indfører en demokratisk styreform går han med det samme bersærk og indleder atomkrige mod alle andre civilisationer han har kontakt med.

Denne temmelig bizarre opførsel skyldes en programfejl---nærmere bestemt skyldes den at to fiktioner (der underligger spillets egen fiktion) ikke stemmer overens. Den ene fiktion er fiktionen om heltal. Den anden er fiktionen om at "talværdierne" i programmeringssproget faktisk repræsenterer heltal.