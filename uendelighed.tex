\chapter{Uendelighed}
Ud fra Peanos beskrivelse af de naturlige tal er det åbenlyst, at \emph{der ikke findes noget største naturlige tal}. Uanset hvor lang en kæde $S(S(S(S(\ldots$ man konstruerer, så er det altid muligt at sætte endnu et $S$ foran---ganske som når små børn finder på store tal, og deres irriterende større søskende driller dem ved altid at finde et større ved at lægge 1 til. Peanos notation beskriver også på denne måde vores typiske forestilling om naturlige tal.

At denne forestilling om tal er en fiktion er tydeligt: Det er muligt at konstruere tal, der ikke tilsvarer nogen kvantitet der faktisk findes i universet: Givet et Peano-tal der beskriver antallet af elementarpartikler i universet kan vi sagtens konstruere et Peano-tal med endnu et $S$ foran. At der ikke er noget største tal betyder at der er uendeligt mange tal---og noget uendeligt stort kan ikke eksistere i et endeligt univers.

Men i matematikkens fiktion \emph{findes} uendelighed.