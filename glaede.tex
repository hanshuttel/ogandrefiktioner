\chapter{Glæden ved fiktionerne}

 Fordi matematikken er en fiktion, er den menneskeskabt og dermed bliver matematikken en fiktion, vi kan skabe sammen. Men der er noget andet særligt ved matematikken: Den er \emph{intersubjektiv} i modsætning til fiktioner som f.eks. poesi eller (i den slemme ende) okkultisme og astrologi mm.
 
 En vigtig grund til at vi bruger ordet "fiktion" i stedet for ordet "model" er at fiktionsbegrebet gør det muligt for os at se forskellige fiktioner som værende lige gyldige, hvor fokus i modelbegrebet er at modellen er en model \emph{af} noget. Man kommer meget let til at begrænse kritikken af en model til en analyse af hvor godt modellen beskriver et stykke virkelighed. På denne måde er virkeligheden primær.
 
 Til gengæld kan en anke mod de forskellige varianter af "platonisme" i matematikfilosofi være det omvendte, nemlig at den gør modellen primær, og virkeligheden sekundær.
 
 Matematik er sådan en god fiktion – ligesom en god roman eller film kan give et forbløffende godt billede af menneskelivet, kan matematikken bruges til at forstå verden med og til at forudsige. 

Sådanne kvaliteter kan god skønlitteratur også have. En velskrevet fortælling er en "litterær model" af menneskets eksistens. Dén analogi er vigtig.

\section{Socialkonstruktivisme?}

Jamen, er det ikke bare socialkonstruktivisme i ny forklædning? Socialkonstruktivismens ærinde er at hævde, at al menneskelig viden er sociale konstruktioner og i sin radikale form er menneskelig viden \emph{ikke andet end} sociale konstruktioner. Kragh \cite{kragh} kritiserer socialkonstruktivismen. Hans indvendinger deler vi.

At de abstrakte objekter, matematikken beskæftiger sig med er fiktive er ikke ensbetydende med at de er \emph{vilkårlige} fiktioner, der udelukkende er opstået på baggrund af kulturelle luner. Selvom det matematiske univers er blevet til under kulturel påvirkning er dets objekter baseret på regelmæssigheder i virkeligheden, der er uafhængige af menneskelig kultur. Hvis en flok på fem får slås sammen med en flok på fire, så vil der være ni får i den nye flok, uanset om de tælles af en før-matematisk stenalderhyrde ved hjælp af tælleperler eller en industrilandmand med et abstrakt symbolsk talbegreb. Det vil ligeledes gælde at fåreflokken herefter kan deles i tre lige store underflokke, men ikke i to. Mange tidlige menneskelige kulturer har uafhængigt af hinanden opdaget disse regelmæssigheder i kvantiteter\cite{heaton}, og disse har dannet udgangspunkt for at indføre fiktive, abstrakte objekter (i eksemplet ville vi skrive dem med symbolerne $4$, $5$ og $9$), der giver redskaber til at arbejde med \emph{alle} tællelige kvantiteter. Andre aspekter af den matematiske fiktion er formet af fællestræk i menneskers kognition og sanseapparat, der går på tværs af alle kulturer.

En (blandt naturvidenskabsfolk) almindelig indvending imod den radikale socialkonstruktivisme er at den er antividenskabelig, og afviser empiriske argumenter. Men et lignende argument er faktisk lige så anvendeligt imod matematisk platonisme! Hvis matematiske objekter har en særlig eksistens, udenfor tid og rum og uden nogen kausal forbindelse til tid og rum, så er de pr. definition udenfor videnskabens diskursunivers.