\chapter{Total fiktion i computeren}

\section{Tal og to-tal}

Når computere skal arbejde med tal, er de nødt til at repræsentere dem på en eller anden måde.

En repræsentation af et tal kalder man for en \emph{numeral}. Numeraler er dermed en fiktion om tal.

Normalt skriver vi i vore dage tal med numeraler fra det, vi kalder for titalssystemet. Tallet sytten skriver vi som $17$, for $17$ er summen af $1$ ti'er og $7$ enere. Tallet femhundredetotogtredive skriver vi som $532$, for $532$ er summen af $5$ hundrede'r, $3$ ti'ere og $2$ enere. Sagt på en anden måde:
%
\[ 532 = 5 \cdot 10^2 + 3 \cdot 10^1 + 2 \cdot 1^0 \]
%
Vi har "opløst" $532$ som en sum af potenser af $10$, hvor koefficienterne er $5$, $3$ og $2$ -- og i alle tilfælde skal være mindre end $10$. Men numeraler kunne også skrives ved at opløse et tal som en sum af potenser af $2$. Dette er totalssystemet, også kendt som det binære talsystem. Her repræsenterer vi tallet sytten som $10001$, for
%
\[ 17 = 1 \cdot 2^4 + 0 \cdot 2^3 +  0 \cdot 2^2 +  0 \cdot 2^1 +  1 \cdot 2^0 \]
%
Her må koefficienterne altså være mindre end $2$, dvs. cifrene $1$ eller $0$. Disse cifre kalder man for \emph{bits} -- "bit" for det engelske "binary digit". 

Hvorfor lige totalssystemet? vil nogen her spørge. Årsagen er enkel: Kredsløbene i en computer er bygget op som milliarder af gates. En gate er en "kontakt" i kredsløbet, der kan være tændt ($1$) eller slukket ($0$
). Derfor kan man repræsentere et tal ved at tænde og slukke for de rigtige kontakter.

\section{Spillet om civilisationen}

I computerspillet \emph{Civilization}, hvor spilleren styrer en fiktiv civilisation fra en stenalderboplads til en rumfarende supermagt, figurerer der fiktive udgaver af virkelighedens historiske verdensledere, hvis adfærd bl.a. reguleres af talværdier - f.eks. er en verdensleders tilbøjelighed til at kaste sig ud i angrebskrige repræsenteret ved en \texttt{aggression}-værdi. Mange af spillets hændelser påvirker de forskellige lederes og kulturers attributter. F.eks. betød indførelsen af demokrati at lederens \texttt{aggression} reduceredes med $2$, for at simulere at demokratier er fredeligere end tyrannier.

En af spillets verdensledere er den indiske revolutionsleder og pacifist, Mohandas Gandhi. For at skildre Gandhi som en fredelig leder valgte spillets udviklere at give ham en \texttt{aggression} på $1$; den laveste hos nogen af spillets fiktionaliserede verdensledere. I hovedparten af spillet opførte Gandhi sig som man ville forvente: Han fokuserede på kulturel og videnskabelig udvikling, etablerede fredelige handelsforbindelser med nabolandene, og forholdt sig neutralt i andres krige. Men han havde også et temmelig uheldigt træk: I det øjeblik han indfører en demokratisk styreform går han med det samme bersærk og indleder atomkrige mod alle andre civilisationer han har kontakt med.

Denne temmelig bizarre opførsel skyldes en programfejl---nærmere bestemt skyldes den at to fiktioner (der underligger spillets egen fiktion) ikke stemmer overens. Den ene fiktion er fiktionen om heltal. Den anden er fiktionen om at "talværdierne" i programmeringssproget, dvs. numeralerne, faktisk repræsenterer heltal.