\chapter{Fiktioner}

Når nu vil påstår, at matematik er en fiktion, er det vigtigt for os først at grundlægge begrebet.

\section{Et ord fra latin}

Ordet \emph{fiktion} kommer selvfølgelig fra latin.

Den Danske Ordbog \cite{ordnet} angiver at ordet kommer

\begin{quote}
   fra latin \textsl{fictio} (genitiv -onis), afledt af latin \textsl{fingere} 'forme, opdigte' 
\end{quote}
 
og giver derefter en af betydningerne som
 
 \begin{quote}
     \ldots antagelse som fremsættes fordi den er hensigtsmæssig som filosofisk eller videnskabelig metode, uagtet den kan være falsk eller modsigelsesfuld
 \end{quote}
 
 Det er netop dén betydning, vi tager udgangspunkt i: At matematikkens begreber er antagelser, skabt af mennesker, fordi de er hensigtsmæssige. Og de vil ofte rumme usandheder, mangler og modsigelser. Arbejdet med at håndtere disse problemer, der er en vigtig del af matematikkens historie.
 
\section{Darth Vaders lyssværd}

Men lad os starte et helt andet sted end i matematikkens verden. Ordet \emph{fiktion} har nemlig også en anden betydning af ordet, nemlig \cite{ordnet}

\begin{quote}
    litteratur (fx roman og novelle), film el.lign. der beskriver helt eller delvis opdigtede personer og begivenheder
\end{quote}

Er påstanden 

\begin{quote}
    Darth Vader kæmper med et rødt lyssværd
\end{quote}
sand?

Det er den strengt taget ikke, for Darth Vader findes ikke, og lyssværd gør heller ikke. Men i den sammenhæng, vi alle sammen accepterer, når vi taler om Star Wars,  er påstanden faktisk sand. Skulle læseren være i tvivl, prøv da at se eller gense nogle af de gamle film. 

Der er et krav, som de to slags fiktioner har fælles, nemlig et krav om at være \emph{modsigelsesfri}. Hvis Darth Vader i nogle scener pludselig kæmpede med et grønt lyssværd, ville vi umiddelbart sige, at det var "forkert", også selv om vi godt ved, at Darth Vader er en fiktion konstrueret af George Lucas.

\section{Pallas Athene}

Eksisterer den græske gudinde Athene? Her vil de allerfleste nulevende mennesker sige nej.

\section{Eksisterer Google?}

Lad os gå lidt længere væk. Eksisterer Google?

For nogle vil svaret være, at ja, virksomheden Google eksisterer. Det gør den så egentlig ikke, thi den hedder faktisk Alphabet. 

Men hvad med søgemaskinen Google?

Hvis Google findes, så tvinges vi til at acceptere at Athene også  fandtes engang. Hun har ihvertfald haft mange ansatte og rådet over store samfundsinstitutioner.

Google og Athene findes, men er fiktioner. Fiktion er ikke er "usand" - mennesker bruger den til at forstå virkeligheden med, og dermed bruger de den også til at forme virkeligheden med. På samme måde med matematik.
Uanset om Athene findes eller ej, så har forestillingen om Athene faktisk formet verden. På samme måde som mange af de fiktive entiteter vi i dag har former verden i dag (og ikke kun i positiv retning).

\section{Tallene}

Tallet 3 findes heller ikke, i den forstand at vi ikke kan gå ud noget sted og kigge på det; der er ingen måleinstrumenter der kan vise os det, og det er ikke en del af tid og rum. Men i matematikkens fiktion findes det, og det har mange egenskaber vi kan tale om.

Men vi kan definere begrebet "tre". Den tyske matematiker og logiker Gottlob Frege talte om "tre" som den egenskab, alle samlinger af størrelsen tre har til fælles.

I næste kapitel vender vi tilbage til tallene.


