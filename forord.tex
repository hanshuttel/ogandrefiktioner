\chapter{Om denne bog}

Denne lille bog handler om eksistens og matematik -- ikke om menneskets eksistens, men om matematiske objekters eksistens eller mangel på samme.

\section{Hvem vi skriver til og hvorfor}

Vores målgruppe er som udgangspunkt alle, der spekulerer på hvad matematik skal gøre godt for. Nogle skal i gang med at lære matematik på en videregående uddannelse. Nogle har haft underlige eller dårlige oplevelser med matematik tidligere, og har måske overbevist sig om at de ikke har ``matematikhjerne''. Nogle er bare nysgerrige.

En af de erfaringer, vi har gjort os, er at en kombination af nogle mærkelige kulturelle forestillinger om matematik og den måde nogle matematikere taler på gør at mange studerende tror at matematik er noget nærmest mystisk og overnaturligt, som ikke rigtig har noget med virkeligheden at gøre. Med til denne tanke hører ofte en forestilling om at forståelse for matematik kræver en helt særlig---muligvis medfødt---gave, som kun et fåtal er i besiddelse af.

\section{Hvad er matematik \emph{egentlig}?}

Vores budskab er, at det forholder sig helt anderledes, end mange regner med. Virkeligheden kom først. Matematikken er en fiktion, vi bruger for at tale om virkeligheden; en fiktion der tillader at ignorere virkelighedens væld af detaljer for bedre at få et mentalt greb om den. Men det er ikke en fiktion der er grebet ud af den blå luft; den fortæller en historie om regelmæssigheder i virkeligheden. Det er mennesker der har udviklet den og bliver ved med at udvikle den---og selvom mange af matematikhistoriens store skikkelser unægteligt var exceptionelt begavede, så kræver matematisk forståelse ikke nogen exceptionel gave. Det kræver bare at man er menneske.

Men hvad er egentlig matematikkens grundlag, hvis matematik er en fiktion? Det har været udgangspunkt for utallige slagsmål blandt matematikere og filosoffer og har dannet grundlag for flere store "ismer" i matematikkens filosofi. Vores morale er, at hele dette slagsmål mellem "ismerne" er -- fiktivt.

De seneste års udvikling inden for datalogi, et fag vi begge færdes i, tyder på at de forskellige "ismer" ikke er konkurrerende religioner, men komplementerende syn på matematikkens grundlag. De kan og skal eksistere samtidig. Det store bevis (om vi så må sige) for at det er sådan, er de computerbaserede bevisassistenter, som nu er ved at vinde frem. Det er datalogien, der er med til at give matematikken dens grundlag nu.

\section{Hvad står der i denne bog?}

I denne bog vil vi give nogle enkle eksempler på de vigtige fiktioner, som findes i matematikken.

Undervejs vil vi vende tilbage til tallene, fordi tallene er en af de vigtigste og mest grundlæggende fiktioner i matematik. Her er vi inspireret af Rantzaus bog \emph{Alle tiders tal} \cite{rantzau}, der er en usædvanligt rig og spændende bog om tallenes historie.